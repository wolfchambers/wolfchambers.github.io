\documentclass[11pt]{article}

\usepackage{jeffe,handout}

\setlength{\headsep}{.5in}

\begin{document}

\headers{CS3200: Programming Languages}{Homework 6}{Spring 2020}

\begin{center}

{\Large\bf Homework 6: Essay on a new language}
\\[.25in]

\end{center}

Experiment  and  learn  a  new  (to  you)  programming  language; examples  might  include  (but are NOT limited to) Matlab, Mathematica, Icon, Ruby, C\#, Fortran, Pascal, Perl, Ada, BASIC,Cobol,  Lisp,  Miranda,  Magma,  Processing,  PHP,  SML,  scheme,  smalltalk,  Erlang,  Io,  Clojure, Squeak, OCamel, etc.  (Note:  Some of these are available freely, but others are not; some are on hopper, and some are not. So you might want to look around for a bit before deciding.)  The only languages specifically forbidden for thisassignment are C, C$++$, Java, Python, Prolog and Haskell, since you are either quite familiar with those after the classes in our major here or will see them in some detail later in this course.  I'll put you on your honor to not choose one you already know, since that defeats the purpose of the assignment!  I hope you'll see this as an opportunity to learn a language that you've been curious about or think would be helpful to know for future projects.

Investigate the features of the language, both through the documentation and writing your own code.  Is the language compiled or interpreted?  Is it functional or imperative or something else? Does the language use lexical or dynamic scoping?  Can scopes nest, and are they open or closed? Does scope encompass an entire block where it is declared?  How are recursive subroutines declaredand used?  Investigate order of evaluation - is it left to right or right to left?  Does the language have short circuit boolean evaluation, or a true boolean type?  Is the language dynamically or statically type-checked,  and  is  it  strong  or  weak  type-checking?   Does  the  language  use  any  coercion  or type-casting functionality, and how is polymorphism incorporated (if at all)?  What type of object oriented  structure  does  it  include,  if  any?   How  are  subroutines  and  exceptions  handled  in  the language? 

Note that you don’t have to address all of these - some aren’t even relevant for all languages.  You're also welcome to describe other interesting or unusual structural features of the language.  My goal is simply to get you to investigate the issues we covered in the first half of the course in more depth. {\bf But:} Be sure to back up your answers, either through relevant and legitimate sourcesor  through  examples  you  devise  to  test  this  yourself.  So your  claims  or  answers  to  those  questions  above should either have a citation or example attached. 

For this assignment, you will write a longer paper; I expect about 4-5 pages for most will coverthe material requested.  I will expect at least 3 {\bf reputable} sources beyond your textbook, with proper citations for relevant facts from those references.  Of course, for this assignment, you are welcome to use web or book resources of any type; just be sure to use at least 3 reputable sources and includethem in your bibliography.  (So you can use wikipedia or reddit if you must, but don’t expect me to  count  it  as  reputable  unless  you  include  code  to  back  up  what  they  claim  on  it!)   For  this assignment, I am also expecting you will likely need to include some brief sample code to justify your answers to some of the questions; please be sure to cite any references you use for this code, but you are also responsible for making sure they actually run and are correct.  You are also welcome to have an appendix with longer code examples, but I expect the main 4-5 pages to be separately coherent.


\end{document}
