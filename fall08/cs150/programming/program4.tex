\documentclass[11pt]{article}
\usepackage{jeffe,handout}

% =========================================================
\begin{document}

\headers{CS 150}{Programming Assignment 4 (due October 22,
2008)}{Fall 2008}

\begin{center}
\LARGE \textbf{CS 150: Intro to OOP, Fall 2008}
\\
\textbf{Programming Assignment 4}
\\[1ex]
\Large Due \emph{via email} by 11:59pm on October 22, 2008\\

\end{center}

For this assignment, you must work individually.  As usual, you are
welcome to discuss {\em general} python syntax, but your design and
coding should all be done individually.  Please refer to the class
syllabus or consult the instructor if you have any questions.

\bigskip

%----------------------------------------------------------------------
%%%%%%%%%%%%%%%%%%%%%%%%%%%%%%%%%%%%%%%%%%%%%%%%%%%
In the first programming assignment, you created an animation which
included several animals. At that time, you relied on the use of the
class, Layer, to manipulate your animal as a single coherent unit,
rather than as a scattered collection of individual shapes. We saw
several advantages in that representation of an animal, such as
being able to easily move the animal in the scene.

That use of the class Layer was a step in the right direction, but
this time we would like to go a step further, having you create and
document a new class which represents your specific animal.

\bigskip

\noindent {\bf Background discussion:}

For the sake of discussion, let's suppose that my first programming
assignment generated an animation which included several monkeys.
Presumably my code included some 30-40 lines of code specifically
for modeling a monkey. I probably instantiated various shapes to
represent the arms, legs and tail of the monkey, as well as
instantiating a new Layer to represent the monkey as a whole. Then
the body parts were added to the layer, and the layer was added to
the canvas. Suppose that I were proud of my representation of a
monkey and that I wanted to add more monkeys into my scene, or
better yet, I wanted to allow other people to conveniently add my
monkeys into their own animations. I might accomplish this by simply
making the 30-40 lines of code which I had used available to others,
verbatim, for inclusion in other places. But a much better design,
in the spirit of object-oriented programming, would be to define a
new class, say Monkey containing the necessary code. Then, others
could use this new class with minimal effort, just as they had used
the more primitive shapes. For example, an artist might simply
specify:

\smallskip

\begin{algorithm}
    chimp = Monkey()\\
    paper.add(chimp)\\
    chimp.move(80,120)\\
    chimp.scratch()
\end{algorithm}

\smallskip

This is our goal. Of course, there is no reason to reinvent the
wheel; we do not wish to create our new class entirely from scratch.
We have already seen that the concept of a Layer is a great model
for representing a Monkey, and so we will use inheritance to define
a Monkey as a subclass of Layer. Therefore, we might start out our
class definition using syntax such as:

\smallskip

\begin{algorithm}
from cs1graphics import *\\
 \\
class Monkey(Layer): \+ \\
\ldots \-
\end{algorithm}

\smallskip

In this way, any object from class Monkey inherits all of the
instance variables and methods associated with the class Layer, such
as move(), draw(), setDepth() and so on. Of course, if we do not add
any additional code to the class definition for Monkey, then our
class will be identical to that of a Layer. This assignment will
require you to add new instance variables and behaviors as well as
overriding existing behaviors.

\bigskip

\newpage

\noindent {\bf Requirements:}

There are three main sets of requirements to keep in mind while
designing your class.

\begin{enumerate}
\item
Create a new class, representing a type of animal, and define at
least three new and distinct methods which are appropriate for
customizing your chosen animal.

In this assignment description, we have been using the discussion of
a class Monkey purely for example. Your project should involve the
development of a new class to represent your own choice of animal,
most likely an animal from your first programming assignment (though
you are free to change your mind).

Please adhere strictly to the following minimum requirements for the
development of such a class:

\begin{itemize}

\item You will need to write a constructor which appropriately
initializes a newly created animal. Presumably, this will involve
the creation and placement of several underlying shapes, which are
then added to the animal using the inherited method add().

You may decide whether or not to allow additional parameters when
calling the constructor which effect the initial settings for your
animal.


\item Think carefully about the geometry from the perspective of the
user. That is, if they execute, chimp.moveTo(100,100), hopefully the
monkey is moved visually to that location with a meaningful
reference point. That is, perhaps the bottom of the foot is moved
there, or the nose, but it should not be that the monkey is played
halfway across the screen. In similar respect, rotate and scale are
based upon the notion of the reference point, so consider how you
define the reference point of one of your instance (by default, it
is at coordinate (0,0) of the Layer).


\item All monkeys are not alike! The point of such a class is not
simply to make precise clones of an animal, but to allow reasonable
variance as well. For example, I might allow a user to change the
eye color of a monkey, the tail thickness, the positioning of the
arms, the direction of the face.  This is where the 3 distinct
methods for customizing your animal come into play.
\end{itemize}

\item Properly document all aspects of your newly defined animal class.

For someone else to know how to use your class, you must provide
sufficient documentation. For this, we would like you to use
docstrings, as described in earlier work. Specifically, please
ensure that:

\begin{itemize}

\item You provide a docstring to begin the class definition, which
gives an overview of the class as a whole.

\item Any method which you create should begin with an appropriate
docstring, giving an overview of the behavior, as well as explicit
descriptions of any parameters or return values.

\item The constructor documentation should give sufficient explanation
of the initial geometry so that a user could properly place on of
these instances into a scene.

\end{itemize}

\item Include a {\em separate} program, called test.py, which tests your animal and
demonstrates several aspects of its use. Your animation for this
assignment does not need to correspond to the precise animation you
created in the original Artist assignment. However, we would like
your new animation to satisfy the following requirements:

\begin{itemize}
\item Demonstrate the use of each new method introduced in your class.

\item Display at least four distinct animals from your new class,
making sure that the properties vary among those animals.
\end{itemize}

\end{enumerate}

\bigskip

\noindent {\bf Extra Credit:}  I'll know it when I see it.

\end{document}
