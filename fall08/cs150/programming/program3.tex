\documentclass[11pt]{article}
\usepackage{jeffe,handout}

% =========================================================
\begin{document}

\headers{CS 150}{Programming Assignment 3 (due October 13,
2008}{Fall 2008}

\begin{center}
\LARGE \textbf{CS 150: Intro to OOP, Fall 2008}
\\
\textbf{Programming Assignment 2}
\\[1ex]
\Large Due \emph{via email} by 11:59pm on October 13, 2008\\

\end{center}

For this assignment, you must work individually.  As usual, you are
welcome to discuss {\em general} python syntax, but your design and
coding should all be done individually.  Please refer to the class
syllabus or consult the instructor if you have any questions.

\begin{problems}
%----------------------------------------------------------------------
%%%%%%%%%%%%%%%%%%%%%%%%%%%%%%%%%%%%%%%%%%%%%%%%%%%
\item (15 points each) Exercises 6.15 and 6.16

Please include both class definitions (one for 6.15 and one for
6.16) in the same .py file.  Make sure to use comments and
meaningful variable names.  Also, remember to test your
implementation before submitting the file!

\item (2 points per part) Extra credit: Create a new class ExtendedSortedSet which
contains the functionality of 6.16 but also has the following
functionality:

\begin{enumerate}
\item Implement an \_\_add\_\_ function (overloading $+$) which takes two sets and returns
the union of them in a new set.

\item Implement an \_\_eq\_\_ function (overloading $==$) which take two sets and returns
True if the sets are the same.

\item Implement an xor(self, other) function, which will be called by
set1.xor(set2).  The function should return  all elements which are
either in set1 or in set2 (but NOT in both).
\end{enumerate}

\end{problems}
\end{document}
