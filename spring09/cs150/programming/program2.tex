\documentclass[11pt]{article}
\usepackage{jeffe,handout}

% =========================================================
\begin{document}

\headers{CS 150}{Programming Assignment 2 (due by 2/6/09 at
11:59pm)}{Fall 2008}

\begin{center}
\LARGE \textbf{CS 150: Intro to OOP, Spring 2009}
\\
\textbf{Programming Assignment 2}
\\[1ex]
\Large Due \emph{via email} by 11:59pm on Friday, February 6, 2008\\

\end{center}

For this assignment, you may work in pairs to complete the work; in
fact, I encourage all of you to find a partner for the assignment.
If you would like a partner but are unable to find one, please
contact the instructor to be paired up with another interested
student.

Exactly one member of the team should email the program file to the
instructor at echambe5 - at - slu.edu by 11:59pm on the date due.
Please remember to include the name of the other person who worked
on the program in the email.

\begin{problems}
%----------------------------------------------------------------------
%%%%%%%%%%%%%%%%%%%%%%%%%%%%%%%%%%%%%%%%%%%%%%%%%%%
\item Exercise 2.37 from the textbook

After reading the description in the textbook, please note the
following things to be careful of in your program:

\begin{itemize}
\item Your program should work for {\em any} string and pattern that
is entered as input.  So don't only solve this for the example and
expect it to work - you'll need to try entering a variety of inputs
to be sure it is actually going to work on all the inputs I could
possibly try!

\item Be sure to test a variety of cases.  Does your code
work if there are multiple copies of the pattern or the reverse? How
about if the reverse comes before the pattern?  What if the two
overlap?  Any of these cases could crash your code if you aren't
careful, so be sure to test them before handing in.

\item You should only replace the {\em first} substring which is
between the pattern and its reverse, not all of them (if the pattern
and reverse happen to appear many times).

\item As always, you are expected to include comments and use
meaningful variable names in your program.

\item Any code which is submitted that does not at least compile -
i.e. I can type ``python program2.py" and enter input strings
without getting errors - will immediately lose 20\% of the points.

\end{itemize}

Also, I strongly recommend that you reread section 2.3.1 on strings
before beginning this assignment, in order to refamiliarize
yourselves with the available methods for this data type.

\end{problems}
\end{document}
