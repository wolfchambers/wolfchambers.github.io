\documentclass[11pt]{article}
\usepackage{jeffe,handout}

% =========================================================
\begin{document}

\headers{CS 150}{Programming Assignment 2 (due January 26 by
11:59pm)}{Fall 2008}

\begin{center}
\LARGE \textbf{CS 150: Intro to OOP, Spring 2009}
\\
\textbf{Programming Assignment 1}
\\[1ex]
\Large Due \emph{via email} by 11:59pm on January 26, 2008\\

\end{center}


For this assignment, you must work individually in regard to the
design and implementation of your project. Please note the
distinction made in our academic integrity policy between general
course material and work which is submitted for this course. We
consider the use of the Python language syntax and the cs1graphics
package in the category of general course material, which you may
discuss freely. However, you must avoid any discussion of code which
is specific to the design of your artwork, submitted for this
assignment. You should not receive direct help from others, nor
should you share your own source code with others.

You will be producing a series of frames on a Canvas object. You may
control the timing of the frames by using the sleep function from
the time module, as described in class and in the textbook.

The theme of the drawings should center around your choice of a
favorite animal. Obviously, with the rather limited selection of
shapes, we expect some of the artwork to be abstract.

Please consider the following list as a mandatory checklist of tools
and techniques you must use. Though we reserve the right to grade
partly based on artistic merit, the majority of the credit for this
assignment will be given based on your meeting the following
criteria.

\begin{itemize}
\item Your animation must have four or more distinct frames.

\item Your first frame should have the background scene which must
include a tree.

\item Another frame should include the addition of
an animal of your choice.

\item Later, a second animal should be included in the scene and that
animal should partially obscure the tree.

\item Eventually, the first animal should be removed from the scene
and the second animal should move locations.

\item Finally, a last frame should be drawn without either animal.

\end{itemize}

You may feel free to intersperse additional frames as you wish.
Also, please make sure that somewhere in your animation, each of the
following techniques are used:

\begin{itemize}

\item You should set the title of the Canvas as an appropriate title
for your artwork (this might also help us discern the type of animal
you chose, in case it is not quite obvious).

\item You must use at least one instance from each of the following
classes: Circle, Rectangle, Polygon, Segment and Text.

\item You must vary the border thickness and colors of some shapes.

\item You must vary the interior colors for some of your FillableShape
objects, including at least one which has transparent interior.

\item Your second animal should be implemented using the class Layer
as outlined in the documentation for the cs1graphics module.

\end{itemize}

For a small amount of extra credit, you may also incorporate some
form of event driven programming, using the wait() method described
in the text. Some creativity is expected here - just turning the sun
red will not get you much extra credit! Also, please be sure to let
me know what the event expected is, particularly if I have to click
on something or enter text.

\end{document}
