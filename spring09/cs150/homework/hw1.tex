\documentclass[11pt]{article}
\usepackage{jeffe,handout}
\def\rmdefault{bch} % Use Charter for main text font.

\def\BOX#1{\fbox{\vbox to #1{\vss\hbox to #1{\hss}}}}
\def\Bigbox{\BOX{0.25in}}
\def\Bigbox{\raisebox{-0.5ex}[0.25in][0pt]{\BOX{0.25in}}}

\hidesolutions

% =========================================================
\begin{document}

\headers{CS 150}{Homework 1 (due February 2 by 11:59pm)}{Spring
2009}

\begin{center}
\LARGE \textbf{CS 150: Intro to Object Oriented Programming, Spring
2009}
\\
\textbf{Homework 1}
\\[1ex]
\Large Due \emph{via email} by 11:59pm on Monday, February 2\\

\end{center}

This homework covers the material from Chapter 2 of the text,
particularly string and lists.  Please read and familiarize yourself
with this material before working on the homework.  In particular,
working through practice problems 2.2, 2.3, 2.15, 2.17, and 2.24
might be helpful for the material on this homework.  You do NOT need
to submit the practice problems for credit - they are merely for
your own benefit.

Please type all answers and email to the instructor at echambe5 - at - slu.edu by 11:59pm on the date due.

\begin{problems}
%----------------------------------------------------------------------

%%%%%%%%%%%%%%%%%%%%%%%%%%%%%%%%%%%%%%%%%%%%%%%%%%%
\item Exercises 2.6 and 2.7 from the text:
\begin{enumerate}

\item Give a series of statements that determine and print the string that
comes second, in alphabetical order, from an initially unsorted
list.

\item Repeat part (a), ensuring that the original list remains unsorted.

\end{enumerate}


%%%%%%%%%%%%%%%%%%%%%%%%%%%%%%%%%%%%%%%%%%%%%%%%%%%
\item Exercise 2.9 from the text:

What is the result of the command $\texttt{range}(13,40,5)$?

%%%%%%%%%%%%%%%%%%%%%%%%%%%%%%%%%%%%%%%%%%%%%%%%%%%
\item Exercise 2.11 from the text:

Suppose that \texttt{numbers} is a list of values.  Give a command
or sequence of commands that outputs the median (or middle) value of
the list.

Note: You may assume that the list has an odd number of values in
it, so that the median is unique and well defined.

%--------------------------------------------------
\item Exercise 2.19 from the text:

Write an expression that is {\bf True} precisely when
\texttt{stringA} is a capitalized version of \texttt{stringB}.

%%%%%%%%%%%%%%%%%%%%%%%%%%%%%%%%%%%%%%%%%%%%%%%%%%%
\item Exercise 2.22 from the text:

Assume that \texttt{person} is a string in the form of `firstName
middleName lastName'.  Give a command or series of commands that
results in the creation of a corresponding string `firstName
middleInitial lastName'.  For example, if $\texttt{person} =$ `Elmer
Joseph Fudd', then the result should be `Elmer J. Fudd'.

%%%%%%%%%%%%%%%%%%%%%%%%%%%%%%%%%%%%%%%%%%%%%%%%%%%%%%
\item Exercise 2.27, parts b, c, e, i, k, and n
\end{problems}

\end{document}
