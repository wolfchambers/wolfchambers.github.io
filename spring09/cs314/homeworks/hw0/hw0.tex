\documentclass[11pt]{article}
\usepackage{jeffe, handout}
\hidesolutions

\setlength{\headsep}{0.5in}

% ======================================================================
\begin{document}

%\thispagestyle{empty}

\headers{CS314: Algorithms}{Homework 0 (due 1/16/2005)}{Spring 2009}

\begin{center}
{\LARGE\bf      CS314: Algorithms}
\\[1ex]
{\Large\bf      Homework 0, due Friday, January 16 at the beginning
of class}
\\[0.25in]
\end{center}

%----------------------------------------------------------------------
\bigskip\hrule\bigskip
\noindent This homework tests your familiarity with the prerequisite
material from Data Structures and Discrete Math, primarily to help
you identify gaps in your knowledge. \textbf{You are responsible for
   filling those gaps on your own.}

%----------------------------------------------------------------------
\bigskip\hrule\bigskip
\noindent Before you do anything else, read the Course Policies on
the webpage.  This web page gives instructions on how to write and
submit homeworks---staple your solutions together in order, write
your name on every page, don't turn in source code, analyze
everything, use good English and good logic, and so forth.

\medskip
\bigskip\hrule\bigskip

\section*{Required Problems}

% ======================================================================
\begin{problems}
    %----------------------------------------------------------------------
    \item \textsc{Recurrences} \hfill (20 points)

    Solve the following recurrences.  State tight asymptotic bounds
    for each function in the form $\Theta(f(n))$ for some recognizable
    function $f(n)$.  You do not need to turn in proofs (in fact,
    please \emph{don't} turn in proofs), but you should do them anyway
    just for practice.  Assume reasonable but nontrivial base cases if
    none are supplied.  More exact solutions are better.
    \begin{enumerate}
        \item $A(n) = 2A(n/2) + \lg n$
        \item $B(n) = 3B(n/2) + n$
        \item $C(n) = 2C(n/2) + n^2$
        \item  $D(n) = 2D(n-1) + 1$
        \item  $E(n) = \max_{1\le k \le n/2} (E(k) + E(n-k) + n)$
        \item $F(n) = 2F(\floor{n/3} + 9) + n^2$
        \item  $G(n) = 2G(n-1)/G(n-2)$
        \item $H(n) = \log( H(n-1) ) + 1$
        \item $I(n) = 2I(\sqrt(n)) + 1$
        \item  $J(n) = J( { n/2 }) + 1$
    \end{enumerate}

\newpage

    \item \textsc{Sorting functions} \hfill (20 points)

    Sort the following 25 functions from asymptotically
    smallest to asymptotically largest, indicating ties if
    there are any.  You do not need to turn in proofs (in
    fact, please \emph{don't} turn in proofs), but you
    should do them anyway just for practice.
    \[
    \begin{array}{c@{\qquad}c@{\qquad}c@{\qquad}c@{\qquad}c}
        1
        &       n
        &       n^2
        &       \lg n
        &       1 + \lg \lg n
        \\[1ex]
        \cos n + 2
        &       n^{\lg n}
        &       (\lg n)!
        &       (\lg n)^{\lg n}
        &       F_n
        \\[1ex] \lg^{1000} n
        &       2^{\lg n}
        &       n \lg n
        &       \Sum_{i=1}^{n} {i}
        &       \Sum_{i=1}^{n} {i^2}
        \\[1ex] n!
        &       \lg (n^{10000})
        &       \floor{\lg \lg (n)}
        &       2^{2 \log n}
        &       15n^2 - 12n + 8 \lg n + 4
    \end{array}
    \]
    To simplify notation, write $f(n) \ll g(n)$ to mean $f(n) =
    o(g(n))$ and $f(n) \equiv g(n)$ to mean $f(n) = \Theta(g(n))$.
    For example, the functions $n^2$, $n$, $\binom{n}{2}$, $n^3$ could
    be sorted either as $n \ll n^2 \equiv \binom{n}{2} \ll n^3$ or as
    $n \ll \binom{n}{2} \equiv n^2 \ll n^3$. \Hint{When considering
       two functions $f(\cdot)$ and $g(\cdot)$ it is sometime useful
       to consider the functions $\ln f(\cdot)$ and $\ln g(\cdot)$.}


    \item \textsc{Trees, Fibonacci numbers, and Induction} \hfill (20 points)

    The $n^{th}$ Fibonacci binary tree $\mathcal{F}_n$ is defined
    recursively as follows:

    \begin{itemize}
    \item $\mathcal{F}_1$ is a single root node with no children.
    \item For all $n \ge2$, $\mathcal{F}_n$ is obtained from
    $\mathcal{F}_{n-1}$ by adding a right child to every leaf node
    and adding a left child to every node that has only one child.
    \end{itemize}

    \begin{enumerate}
    \item Prove that the number of leaves in $\mathcal{F}_n$ is
    precisely the $n^{th}$ Fibonacci number: $F_0 = 0, F_1 = 1,$ and
    $F_n = F_{n-1} + F_{n-2}$ for all $n \ge 2$.

    \item How many nodes does $\mathcal{F}_n$ have?  For full
    credit, give an \emph{exact} closed form answer in terms of the
    Fibonacci numbers and prove that your answer is correct.

    \item Prove that for $n \ge 2$, the left subtree of $\mathcal{F}_n$ is a copy
    of $\mathcal{F}_{n-2}$. (Hint: This is easier than it sounds!)
    \end{enumerate}

    \item \textsc{Fractions and Pigeonholes} \hfill (20 points)

    The \emph{fractional part} of $x$ is the amount by which $x$
    exceeds its floor, $\floor{x}$.  For $x \in \R$ and $n \in \N$,
    let $S = \set{x, 2x, \ldots, (n-1)x}$.

    \begin{enumerate}
        \item Prove that if some pair of numbers in $S$ have
        fractional parts that differ by at most $1/n$, then some
        number in $S$ is within $1/n$ of an integer.

        \item Use part (a) to prove that some number in $S$ is
        within $1/n$ of an integer.

    \end{enumerate}

\end{problems}

\end{document}
